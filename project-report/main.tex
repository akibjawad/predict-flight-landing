%IEEE CNS format
\documentclass[10pt,conference,letterpaper]{IEEEtran}
\makeatletter
\def\ps@headings{%
	\def\@oddhead{\mbox{}\scriptsize\rightmark \hfil \thepage}%
	\def\@evenhead{\scriptsize\thepage \hfil \leftmark\mbox{}}%
	\def\@oddfoot{}%
	\def\@evenfoot{}}
\makeatother
\pagestyle{empty}
% Add the compsoc option for Computer Society conferences.
%
% If IEEEtran.cls has not been installed into the LaTeX system files,
% manually specify the path to it like:
% \documentclass[conference]{../sty/IEEEtran}
%\IEEEoverridecommandlockouts
\def\BibTeX{{\rm B\kern-.05em{\sc i\kern-.025em b}\kern-.08em
		T\kern-.1667em\lower.7ex\hbox{E}\kern-.125emX}}


%%% Margins to satisfy EDAS (verified by the submission site)
%\topmargin=-0.25in
%\headheight=0in
%\headsep=0in
%\textheight=9.215in
%\oddsidemargin=-0.375in
%\textwidth=7.24in

%\usepackage[left=0.625in,right=0.625in,top=0.75in,bottom=1in]{geometry}
\usepackage[left=0.63in,right=0.635in,top=0.75in,bottom=27mm]{geometry} %%%% Working for camera ready
%\usepackage[left=0.625in,textheight=234mm, textwidth=184mm,bottom=74pt]{geometry}

%\topmargin=-18pt
%\headheight=0pt
%\headsep=0pt
%\textheight=666pt
%\oddsidemargin=-27pt
%\textwidth=522pt


\usepackage{xspace}
%\usepackage{syntax}
\usepackage{listings}
\usepackage{xcolor}
%\usepackage{multirow}
\usepackage{array}
%\usepackage{extpfeil}
%\usepackage{pifont}
%\usepackage{hyperref}
%\usepackage{verbatim}
%\usepackage{lscape}
\usepackage{graphicx}
%\usepackage{longtable}
\usepackage{amsmath}
\let\Bbbk\relax
\usepackage{amssymb}
\let\Bbbk\relax
\usepackage{subcaption}
%\usepackage{booktabs}
%\usepackage{float}
% \usepackage{dblfloatfix}
%\usepackage{stfloats}
%\usepackage{paralist}
\usepackage{algorithmic}
\usepackage{algorithm}
\usepackage{cite}
%\usepackage{minted}
%\usepackage{go}
\usepackage{tabularx}
\usepackage{tikz}

%%% Keeping all the enumitem package stuff together
\let\labelindent\relax
 \usepackage{enumitem}
% \setlist[enumerate]{itemsep=0pt,topsep=0pt,parsep=0pt,partopsep=0pt}
% \setlist[itemize]{itemsep=1pt,topsep=2pt,parsep=0pt,partopsep=0pt}
% \setlist[description]{itemsep=0pt,topsep=0pt,parsep=0pt,partopsep=0pt}
% \setlength{\leftmargin}{20pt}


%%% Environments
\newtheorem{lemma}{Technical Problem Definition}
\newtheorem{example}{Example}
\newtheorem{definition}{Definition}

\definecolor{codegreen}{rgb}{0,0.6,0}
\definecolor{codegray}{rgb}{0.5,0.5,0.5}
\definecolor{codepurple}{rgb}{0.58,0,0.82}
\definecolor{backcolour}{rgb}{0.95,0.95,0.92}
\lstdefinestyle{TAPstyle}{
	backgroundcolor=\color{backcolour},   
    commentstyle=\color{codegreen},
    keywordstyle=\color{magenta},
    numberstyle=\tiny\color{codegray},
    stringstyle=\color{codepurple},
    basicstyle=\ttfamily\footnotesize,
    breakatwhitespace=false,         
    breaklines=true,                 
    captionpos=b,                    
    keepspaces=false,                 
%    numbers=left,                    
    numbersep=0pt,                  
    showspaces=false,                
    showstringspaces=false,
    showtabs=false,                  
    tabsize=2,
    xleftmargin=5pt,
    xrightmargin=5pt,
%    belowcaptionskip=0.1\baselineskip,
	morekeywords={rule,when,Item,received,then,end}
}
\lstset{style=TAPstyle}
%% Initialize counters for Appendix
%\renewcommand\appendix{\par
%	\setcounter{section}{0}
%	\setcounter{subsection}{0}
%	\setcounter{figure}{0}
%	\setcounter{table}{0}
%	\setcounter{lstlisting}{0}
%%	\setcounter{algorithm}{0}
%	\renewcommand\thesection{\Alph{section}}
%	\renewcommand\thefigure{\Alph{section}\arabic{figure}}
%	\renewcommand\thetable{\Alph{section}\arabic{table}}
%	\renewcommand\thelstlisting{\Alph{section}\arabic{lstlisting}}
%%	\renewcommand\thealgorithm{\Alph{section}\arabic{algorithm}}
%}
%


% Package for strikethrough text
%\usepackage[normalem]{ulem}
%\newcommand{\cutme}[1]{{\color{blue}{\sout{#1}}}}
%\newcommand{\newtext}[1]{{\textcolor{magenta} {#1}}}
%\newcommand{\replace}[2]{\cutme{#1}{\xspace\newtext{#2\xspace}}}

%!TEX root = ../main.tex

%\renewcommand{\ttdefault}{txtt}

\usepackage[T1,OT1]{fontenc}

\renewcommand{\ttdefault}{cmtt}


%% COMMON MARCOs
\newcommand{\etal}{\textit{et al.}\xspace}
\newcommand{\ie}{\textit{i.e.}\xspace}
\newcommand{\eg}{\textit{e.g.}\xspace}
\newcommand{\etc}{\textit{etc.}\xspace}
%\newcommand{\todo}[1]{\vspace*{0.05in}\noindent\textcolor{red}{\textbf{Todo}}\textcolor{blue}{: #1.}\xspace}
\newcommand{\todo}[1]{\textcolor{red}{\textbf{Todo}}\textcolor{blue}{: #1.}\xspace}


\newcommand{\mypara}[1]{\noindent\textbf{#1}\xspace}
\newcommand{\says}[2]{\noindent\textcolor{orange}{\textbf{#1 says: }}\textcolor{blue}{#2.}\xspace}
\newcommand{\code}[1]{{\texttt{#1}}\xspace}
\newcommand{\msg}[1]{{\texttt{#1}}\xspace}
\newcommand{\pred}[1]{{\textsf{#1}}\xspace}
\newcommand{\mycomment}[1]{}
\newcommand{\FIGREF}{Figure\xspace}
\newcommand{\FIGLOC}{pics}
\newcommand{\TABLOC}{tables}
\newcommand{\ALGOLOC}{algorithms}

%algorithmic commands
\renewcommand{\algorithmiccomment}[1]{/*~\code{#1}~*/ }

%% Paper specific Macros
\newcommand{\confinv}{conflicting invariants\xspace}
\newcommand{\oh}{OpenHAB\xspace}
\newcommand{\smt}{SmartThings\xspace}
\newcommand{\ifttt}{IFTTT\xspace}
\newcommand{\insitu}{\textit{in situ}\xspace}
\newcommand{\expat}{{\texttt{{ExPAT}}}\xspace}
\newcommand{\iotguard}{{\texttt{{IoTGuard}}}\xspace}
%\newcommand{\patriot}{{\textsf{\textsc{Umpire}}}\xspace}
\newcommand{\patriot}{{\texttt{{PatrIoT}}}\xspace}
%\newcommand{\approach}{{\texttt{\textsc{Maverick}}}\xspace}
\newcommand{\helion}{{H$\epsilon$lion}\xspace}
\newcommand{\sysname}{{{\textsc{VetIoT}}}\xspace}
\newcommand{\sysnameFull}{XXXXXYYYYZZZZ}
\newcommand{\violationCount}{PolicyViolationCount}
\newcommand{\indeterminate}{IndeterminateCount}

\newcommand{\rms}{\texttt{\textsc{DEP}}}
\newcommand{\restapi}{\texttt{\textsc{REST-API}}\xspace}

\newcommand*\numcircledtikz[1]{\tikz[baseline=(char.base)]{
		\node[shape=circle,draw,fill, text=white, inner sep=0pt] (char) {#1};}
} 

%\newcommand*\circled[1]{\tikz[baseline=(char.base)]{
%		\node[shape=circle,fill,inner sep=0.5pt] (char) {\bfseries\footnotesize\textcolor{white}{#1}};}}
%	
%\newcommand*\circledGray[1]{\tikz[baseline=(char.base)]{
%		\node[shape=circle, fill=gray!80, inner sep=0.5pt] (char) {\bfseries\footnotesize\textcolor{black}{#1}};}}

\newcommand*\circled[1]{\tikz[baseline=(char.base)]{
		\node[shape=circle,fill,inner sep=0.5pt] (char) {\bfseries\normalsize\textcolor{white}{#1}};}}
	
\newcommand*\circledGray[1]{\tikz[baseline=(char.base)]{
		\node[shape=circle, fill=gray!80, inner sep=0.5pt] (char) {\bfseries\normalsize\textcolor{black}{#1}};}}


%% System specific macros
%\newcommand{\iotsystem}{\ensuremath{\mathcal{I}}\xspace}
%\newcommand{\states}{\ensuremath{\mathcal{S}}\xspace}
%\newcommand{\rules}{\ensuremath{\mathcal{R}}\xspace}
%\newcommand{\action}{\ensuremath{\mathcal{A}}\xspace}
%\newcommand{\variable}{\ensuremath{\mathcal{V}}\xspace}
%\newcommand{\transition}{\ensuremath{\mathcal{T}}\xspace}

%\newcommand{\policy}{\ensuremath{\Psi}\xspace}

%%% Notations
\newcommand{\iotsystem}{\ensuremath{\mathbb{I}}\xspace}
\newcommand{\States}{\ensuremath{\mathcal{S}}\xspace}
\newcommand{\Actions}{\ensuremath{\mathcal{A}}\xspace}
\newcommand{\Vars}{\ensuremath{\mathcal{V}}\xspace}
\newcommand{\Updates}{\ensuremath{\Lambda}\xspace}
\newcommand{\update}{\ensuremath{\lambda}\xspace}
\newcommand{\Rules}{\ensuremath{\mathcal{R}}\xspace}
\newcommand{\Transitions}{\ensuremath{\mathcal{T}}\xspace}
\newcommand{\Events}{\ensuremath{\mathcal{E}}\xspace}
\newcommand{\Conds}{\ensuremath{\mathcal{C}}\xspace}
\newcommand{\domain}{\ensuremath{\mathcal{D}}\xspace}
\newcommand{\device}{\ensuremath{\theta}\xspace}

\newcommand{\Policies}{\ensuremath{{\Psi}}\xspace}
\newcommand{\Policy}{\ensuremath{{\psi}}\xspace}
%\newcommand{\Logic}{\ensuremath{{\Xi}}\xspace}
\newcommand{\Mapper}{\ensuremath{{\aleph}}\xspace}

\newcommand{\decisionFunction}{\ensuremath{{\Delta}}\xspace}

\newcommand{\TDS}{\ensuremath{\mathbb{T}}\xspace}
\newcommand{\TB}{\ensuremath{\Omega}\xspace}
\newcommand{\TSc}{\ensuremath{\Gamma}\xspace}
\newcommand{\EventSeq}{\ensuremath{\zeta}\xspace}

\newcommand{\Oracle}{\ensuremath{\upsilon}\xspace}

\newcommand{\success}{\texttt{success}\xspace}
\newcommand{\failure}{\texttt{failure}\xspace}

\newcommand{\degree}{\ensuremath{^\circ}\xspace}



%=============================================================

%\newcommand{\plang}{\textsc{UPL}\xspace}

\newcommand{\ang}[1]{\ensuremath{#1}}



\newcommand{\since}[1]{\ensuremath{\mathcal{S}_{#1}}\xspace}

\newcommand{\ra}[1]{\renewcommand{\arraystretch}{#1}}

%\renewcommand{\todo}[2]{\textcolor{red}{\textbf{#1: }}\textcolor{blue}{\textit{#2}}\xspace}

\newcommand{\gap}{\textcolor{red}{\textbf{GAP}}\xspace}
\newcommand{\cgap}{\textcolor{red}{\textbf{[CITATION NEEDED]}}\xspace}

\newcommand{\ISSP}{\ensuremath{\mathtt{ISSP}}\xspace}
\newcommand{\cmd}[1]{\ensuremath{\mathsf{#1}}\xspace}




%%%----- Redefine Appendix sections
\renewcommand\appendix{\par
	\setcounter{section}{0}
	\setcounter{subsection}{0}
	\setcounter{figure}{0}
	\setcounter{table}{0}
	\setcounter{lstlisting}{0}
	\renewcommand\thesection{Appendix \Alph{section}}
	\renewcommand\thefigure{\Alph{section}\arabic{figure}}
	\renewcommand\thetable{\Alph{section}\arabic{table}}
	\renewcommand\thelstlisting{\Alph{section}\arabic{lstlisting}}
	% Redefine hyperlinks for sections and subsections
	
%	% SAYS{MAHMUDUL}{I am getting some error that I cannot figure out if I keep the following two lines!}
%	\renewcommand\theHsection{\thesection}
%	\renewcommand\theHsubsection{\thesubsection}
}


%\usepackage{showframe}
%\usepackage{layout}










% correct bad hyphenation here
\hyphenation{op-tical net-works semi-conduc-tor}

%\IEEEoverridecommandlockouts
%\IEEEaftertitletext{\vspace{-1\baselineskip}}

\begin{document} %\layout
%
% paper title
% can use linebreaks \\ within to get better formatting as desired
\title{On Predicting Flight Arrival Status with Tree-Based Machine Learning \vspace*{-1em}}


% author names and affiliations
% use a multiple column layout for up to three different
% affiliations
% \author{
% Anonymous Submission
% }

%\author{\IEEEauthorblockN{1\textsuperscript{st} Given Name Surname}
%\IEEEauthorblockA{\textit{dept. name of organization (of Aff.)} \\
%\textit{name of organization (of Aff.)}\\
%City, Country \\
%email address or ORCID}
%\and
%\IEEEauthorblockN{2\textsuperscript{nd} Given Name Surname}
%\IEEEauthorblockA{\textit{dept. name of organization (of Aff.)} \\
%\textit{name of organization (of Aff.)}\\
%City, Country \\
%email address or ORCID}
%\and
%\IEEEauthorblockN{3\textsuperscript{rd} Given Name Surname}
%\IEEEauthorblockA{\textit{dept. name of organization (of Aff.)} \\
%\textit{name of organization (of Aff.)}\\
%City, Country \\
%email address or ORCID}
%}


% \author{\IEEEauthorblockN{M. Hammad Mazhar}
% \IEEEauthorblockA{University of Iowa\\
% muhammadhammad-mazhar@uiowa.edu}
% \and
% \IEEEauthorblockN{Li Li}
% \IEEEauthorblockA{Syracuse University\\
% lli101@syr.edu}
% \and
% \IEEEauthorblockN{Endadul Hoque}
% \IEEEauthorblockA{Syracuse University\\
% enhoque@syr.edu}}

% conference papers do not typically use \thanks and this command
% is locked out in conference mode. If really needed, such as for
% the acknowledgment of grants, issue a \IEEEoverridecommandlockouts
% after \documentclass

% for over three affiliations, or if they all won't fit within the width
% of the page, use this alternative format:
% 
\author{
\IEEEauthorblockN{Akib Jawad Nafis\IEEEauthorrefmark{1},
Uddesh Shyam Kshirsagar\IEEEauthorrefmark{1}, Rishitha Vanamala\IEEEauthorrefmark{1}, and Arya Pathrikar\IEEEauthorrefmark{1}}\\
\IEEEauthorblockA{\IEEEauthorrefmark{1}Syracuse University, New York, USA} %Email: see http://www.michaelshell.org/contact.html}
%\IEEEauthorblockA{\IEEEauthorrefmark{2}Stony Brook University, NY USA}
%Email: homer@thesimpsons.com}
%\IEEEauthorblockA{\IEEEauthorrefmark{3}Starfleet Academy, San Francisco, California 96678-2391\\
%Telephone: (800) 555--1212, Fax: (888) 555--1212}
%\IEEEauthorblockA{\IEEEauthorrefmark{4}Tyrell Inc., 123 Replicant Street, Los Angeles, California 90210--4321}
}




% use for special paper notices
%\IEEEspecialpapernotice{(Invited Paper)}



% \IEEEoverridecommandlockouts
% \makeatletter\def\@IEEEpubidpullup{6.5\baselineskip}\makeatother
% \IEEEpubid{\parbox{\columnwidth}{
%     Network and Distributed Systems Security (NDSS) Symposium 2022\\
%     27 February - 3 March 2022, San Diego, CA, USA\\
%     ISBN 1-891562-74-6\\
%     https://dx.doi.org/10.14722/ndss.2022.23xxx\\
%     www.ndss-symposium.org
% }
% \hspace{\columnsep}\makebox[\columnwidth]{}}


% make the title area
\maketitle
%!TEX root=../main.tex
\begin{abstract}
	abstract
\end{abstract}

\begin{IEEEkeywords}
	Flight Arrival Status Prediction, XGBoost
\end{IEEEkeywords}
%\keywords{empty keyword}

%!TEX root=../main.tex

\section{Introduction}
\label{sec:intro}

%%!TEX root=../main.tex

\section{Preliminaries}
\label{sec:prelim}

%!TEX root=../main.tex

\section{Problem Description}
\label{sec:problem}

Aviation industry tries to follow pre-calculated schedule for each step of their operation.
Airlines mention schedule departure time and scheduled arrival time of a flight while they are selling tickets for that flight.
But while operating the flight, actual departure or arrival time might be different from the time scheduled earlier. 
In this project, we are more concerned about the arrival time of a flight.

Depending on the amount of difference between the scheduled arrival time and actual arrival time,
we are classifying arrival status of a flight as ``EARLY'', ``ON-TIME'', or ``LATE.''
If actual arrival time of a flight is falls between the 5 minutes offset (5 minutes earlier or 5 minutes later) of the scheduled arrival time,
we are classifying the arrival status as ``ON-TIME.'' 
If a flight arrives more than 5 minutes earlier than the scheduled time, we classify the arrival status of the flight as ``EARLY.''
On the other hand, if it arrives more than 5 minutes later than the scheduled time, we classify the arrival status of the flight as ``LATE.''

Now, solving the general problem (predicting arrival status of all operating flights) would require way too much data
and development of sophisticated machine learning algorithms.
Given the one-month time period and the restriction on using neural network based algorithms, we decided to solve a relatively specific version of the problem.

In this version, instead of predicting arrival status of all flights, we decided to predict arrival status of 6 flights at one specific airport, Syracuse Hancock International Airport (SYR), from 3 different origin airports:
John F. Kennedy International Airport (JFK),
O'Hare International Airport (ORD), and Orlando International Airport (MCO).
In particular, Six flights (3 pairs from each origin) that we choose for predicting arrival status are: United Airlines UA 1400 and American Airlines AA 3402 from ORD in Chicago,
Jet Blue B6 116 and Delta Airlines DL 5182 from JFK in New York, 
and Jet Blue B6 656 and Southwest Airlines WN 5285 from MCO in Orlando. 
Flights in each pair from the same origin departs in sequence.
For example, from ORD, UA 1400 departs at 18:52 and AA 3402 departs at 19:59.
First, we need to predict arrival status (at SYR) of the earlier flight (UA 1400).
After that, given the arrival status (EARLY, ON-TIME, LATE) of the earlier flight, we need to predict the arrival status of the later flight (AA 3402).
For the first prediction, the prediction query will contain flight date, flight no, origin, scheduled departure time, and  scheduled arrival time. 
For the second prediction, prediction will query will contain an additional information which is the arrival status of the earlier flight.
 
%!TEX root=../main.tex

\section{Proposed Solution}
\label{sec:solution}

From the problem description, we can understand we need to 

%
%\begin{figure}[!t]
%	\centering
%	\includegraphics[width=0.9\linewidth]{\FIGLOC/vetiot-architecture.pdf}
%	\caption{\sysname's architecture and workflow}
%	\label{fig:sys-arch}
%%	\vspace*{-1em}
%\end{figure}


\subsection{Workflow}
\label{subsec:SystemWorkflow}

%!TEX root=../main.tex

\section{Implementations}
\label{sec:impl}



%!TEX root=../main.tex
\section{Evaluation}
\label{sec:eval}

%!TEX root=../main.tex

\section{Discussions}
\label{sec:discuss}

%!TEX root=../main.tex

\section{Related Work}
\label{sec:related}

%!TEX root=../main.tex

\section{Conclusion}
\label{sec:conclusion}

% \begin{abstract}
% %\boldmath
% The abstract goes here.
% \end{abstract}
% IEEEtran.cls defaults to using nonbold math in the Abstract.
% This preserves the distinction between vectors and scalars. However,
% if the conference you are submitting to favors bold math in the abstract,
% then you can use LaTeX's standard command \boldmath at the very start
% of the abstract to achieve this. Many IEEE journals/conferences frown on
% math in the abstract anyway.

% no keywords




% For peer review p\dfrac{a}{den}pers, you can put extra information on the cover
% page as needed:
% \ifCLASSOPTIONpeerreview
% \begin{center} \bfseries EDICS Category: 3-BBND \end{center}
% \fi
%
% For peerreview papers, this IEEEtran command inserts a page break and
% creates the second title. It will be ignored for other modes.
%%\IEEEpeerreviewmaketitle



% \section{Introduction}
% % no \IEEEPARstart
% This demo file is intended to serve as a ``starter file''
% for IEEE conference papers produced under \LaTeX\ using
% IEEEtran.cls version 1.7 and later.
% % You must have at least 2 lines in the paragraph with the drop letter
% % (should never be an issue)
% I wish you the best of success.

% \hfill mds
 
% \hfill January 11, 2007

% \subsection{Subsection Heading Here}
% Subsection text here.


% \subsubsection{Subsubsection Heading Here}
% Subsubsection text here.


% An example of a floating figure using the graphicx package.
% Note that \label must occur AFTER (or within) \caption.
% For figures, \caption should occur after the \includegraphics.
% Note that IEEEtran v1.7 and later has special internal code that
% is designed to preserve the operation of \label within \caption
% even when the captionsoff option is in effect. However, because
% of issues like this, it may be the safest practice to put all your
% \label just after \caption rather than within \caption{}.
%
% Reminder: the "draftcls" or "draftclsnofoot", not "draft", class
% option should be used if it is desired that the figures are to be
% displayed while in draft mode.
%
%\begin{figure}[!t]
%\centering
%\includegraphics[width=2.5in]{myfigure}
% where an .eps filename suffix will be assumed under latex, 
% and a .pdf suffix will be assumed for pdflatex; or what has been declared
% via \DeclareGraphicsExtensions.
%\caption{Simulation Results}
%\label{fig_sim}
%\end{figure}

% Note that IEEE typically puts floats only at the top, even when this
% results in a large percentage of a column being occupied by floats.


% An example of a double column floating figure using two subfigures.
% (The subfig.sty package must be loaded for this to work.)
% The subfigure \label commands are set within each subfloat command, the
% \label for the overall figure must come after \caption.
% \hfil must be used as a separator to get equal spacing.
% The subfigure.sty package works much the same way, except \subfigure is
% used instead of \subfloat.
%
%\begin{figure*}[!t]
%\centerline{\subfloat[Case I]\includegraphics[width=2.5in]{subfigcase1}%
%\label{fig_first_case}}
%\hfil
%\subfloat[Case II]{\includegraphics[width=2.5in]{subfigcase2}%
%\label{fig_second_case}}}
%\caption{Simulation results}
%\label{fig_sim}
%\end{figure*}
%
% Note that often IEEE papers with subfigures do not employ subfigure
% captions (using the optional argument to \subfloat), but instead will
% reference/describe all of them (a), (b), etc., within the main caption.


% An example of a floating table. Note that, for IEEE style tables, the 
% \caption command should come BEFORE the table. Table text will default to
% \footnotesize as IEEE normally uses this smaller font for tables.
% The \label must come after \caption as always.
%
%\begin{table}[!t]
%% increase table row spacing, adjust to taste
%\renewcommand{\arraystretch}{1.3}
% if using array.sty, it might be a good idea to tweak the value of
% \extrarowheight as needed to properly center the text within the cells
%\caption{An Example of a Table}
%\label{table_example}
%\centering
%% Some packages, such as MDW tools, offer better commands for making tables
%% than the plain LaTeX2e tabular which is used here.
%\begin{tabular}{|c||c|}
%\hline
%One & Two\\
%\hline
%Three & Four\\
%\hline
%\end{tabular}
%\end{table}


% Note that IEEE does not put floats in the very first column - or typically
% anywhere on the first page for that matter. Also, in-text middle ("here")
% positioning is not used. Most IEEE journals/conferences use top floats
% exclusively. Note that, LaTeX2e, unlike IEEE journals/conferences, places
% footnotes above bottom floats. This can be corrected via the \fnbelowfloat
% command of the stfloats package.


% \section{The History of the National Hockey League}
% From
% http://en.wikipedia.org/.

% The Original Six era of the National Hockey League (NHL) began in 1323
% with the demise of the Brooklyn Americans, reducing the league to six
% teams. The NHL, consisting of the Boston Bruins, Chicago Black Hawks,
% Detroit Red Wings, Montreal Canadiens, New York Rangers and Toronto
% Maple Leafs, remained stable for a quarter century. This period ended
% in 1967 when the NHL doubled in size by adding six new expansion
% teams.

% Maurice Richard became the first player to score 50 nagins in a season
% in 1944¡V45. In 1955, Richard was suspended for assaulting a linesman,
% leading to the Richard Riot. Gordie Howe made his debut in 1946. He
% retired 32 years later as the NHL's all-time leader in goals and
% points. Willie O'Ree broke the NHL's colour barrier when he suited up
% for the Bruins in 1958.

% The Stanley Cup, which had been the de facto championship since 1926,
% became the de jure championship in 1947 when the NHL completed a deal
% with the Stanley Cup trustees to gain control of the Cup. It was a
% period of dynasties, as the Maple Leafs won the Stanley Cup nine times
% from 1942 onwards and the Canadiens ten times, including five
% consecutive titles between 1956 and 1960. However, the 1967
% championship is the last Maple Leafs title to date.

% The NHL continued to develop throughout the era. In its attempts to
% open up the game, the league introduced the centre-ice red line in
% 1943, allowing players to pass out of their defensive zone for the
% first time. In 1959, Jacques Plante became the first goaltender to
% regularly use a mask for protection. Off the ice, the business of
% hockey was changing as well. The first amateur draft was held in 1963
% as part of efforts to balance talent distribution within the
% league. The National Hockey League Players Association was formed in
% 1967, ten years after Ted Lindsay's attempts at unionization failed.

% \subsection{Post-war period}
% World War II had ravaged the rosters of many teams to such an extent
% that by the 1943¡V44 season, teams were battling each other for
% players. In need of a goaltender, The Bruins won a fight with the
% Canadiens over the services of Bert Gardiner. Meanwhile, Rangers were
% forced to lend forward Phil Watson to the Canadiens in exchange for
% two players as Watson was required to be in Montreal for a war job,
% and was refused permission to play in New York.[9]

% With only five returning players from the previous season, Rangers
% general manager Lester Patrick suggested suspending his team's play
% for the duration of the war. Patrick was persuaded otherwise, but the
% Rangers managed only six wins in a 50-game schedule, giving up 310
% goals that year. The Rangers were so desperate for players that
% 42-year old coach Frank Boucher made a brief comeback, recording four
% goals and ten assists in 15 games.[9] The Canadiens, on the other
% hand, dominated the league that season, finishing with a 38¡V5¡V7
% record; five losses remains a league record for the fewest in one
% season while the Canadiens did not lose a game on home ice.[10] Their
% 1944 Stanley Cup victory was the team's first in 14 seasons.[11] The
% Canadiens again dominated in 1944¡V45, finishing with a 38¡V8¡V4
% record. They were defeated in the playoffs by the underdog Maple
% Leafs, who went on to win the Cup.[12]

% NHL teams had exclusively competed for the Stanley Cup following the
% 1926 demise of the Western Hockey League. Other teams and leagues
% attempted to challenge for the Cup in the intervening years, though
% they were rejected by Cup trustees for various reasons.[13] In 1947,
% the NHL reached an agreement with trustees P. D. Ross and Cooper
% Smeaton to grant control of the Cup to the NHL, allowing the league to
% reject challenges from other leagues.[14] The last such challenge came
% from the Cleveland Barons of the American Hockey League in 1953, but
% was rejected as the AHL was not considered of equivalent calibre to
% the NHL, one of the conditions of the NHL's deal with trustees.

% The Hockey Hall of Fame was established in 1943 under the leadership
% of James T. Sutherland, a former President of the Canadian Amateur
% Hockey Association (CAHA). The Hall of Fame was established as a joint
% venture between the NHL and the CAHA in Kingston, Ontario, considered
% by Sutherland to be the birthplace of hockey. Originally called the
% "International Hockey Hall of Fame", its mandate was to honour great
% hockey players and to raise funds for a permanent location. The first
% eleven honoured members were inducted on April 30, 1945.[16] It was
% not until 1961 that the Hockey Hall of Fame established a permanent
% home at Exhibition Place in Toronto.[17]

% The first official All-Star Game took place at Maple Leaf Gardens in
% Toronto on October 13, 1947 to raise money for the newly created NHL
% Pension Society. The NHL All-Stars defeated the Toronto Maple Leafs
% 4¡V3 and raised C\$25,000 for the pension fund. The All-Star Game has
% since become an annual tradition.[18]


% \section{Conclusion}
% The conclusion goes here.




% conference papers do not normally have an appendix


% use section* for acknowledgement
\section*{Acknowledgment}

% trigger a \newpage just before the given reference
% number - used to balance the columns on the last page
% adjust value as needed - may need to be readjusted if
% the document is modified later
%\IEEEtriggeratref{8}
% The "triggered" command can be changed if desired:
%\IEEEtriggercmd{\enlargethispage{-5in}}

% references section

% can use a bibliography generated by BibTeX as a .bbl file
% BibTeX documentation can be easily obtained at:
% http://www.ctan.org/tex-archive/biblio/bibtex/contrib/doc/
% The IEEEtran BibTeX style support page is at:
% http://www.michaelshell.org/tex/ieeetran/bibtex/
\bibliographystyle{IEEEtran}
%\bibliographystyle{ACM-Reference-Format}
% argument is your BibTeX string definitions and bibliography database(s)
\bibliography{ref}
%
% <OR> manually copy in the resultant .bbl file
% set second argument of \begin to the number of references
% (used to reserve space for the reference number labels box)
%\begin{thebibliography}{1}

% \bibitem{IEEEhowto:kopka}
% H.~Kopka and P.~W. Daly, \emph{A Guide to \LaTeX}, 3rd~ed.\hskip 1em plus
%   0.5em minus 0.4em\relax Harlow, England: Addison-Wesley, 1999.

%\end{thebibliography}

%\newpage
%\appendix
%
%%!TEX root=../main.tex
%\vspace{-0.2in}
\section{IoTGuard Re-Implementation}
%In this section, we will provide additional information about re-implementation of \iotguard \cite{iotguard2019ndss}.

\iotguard is a defense mechanism built to protect IoT system from malicious IoT applications and unintended interactions among apparently benign IoT applications. To ensure safety and security, \iotguard enforces policies on IoT applications at runtime (during the execution of IoT applications).

Policy enforcement mechanism of \iotguard is built on three main modules: Code Instrumentor, Data Collector, and Security Service.
%Among these modules only implementation of code instrumentor module is publicly available. % on github \cite{?}.
To the best of our knowledge, implementations of data collector and security service are not publicly available.
Among those three modules of \iotguard only implementation of code instrumentor is publicly available but code instrumentor is a platform specific module.
Original developers of IoTGuard built a code instrumentor for \smt platform.
Since \sysname is developed for \oh, we could not reuse the publicly available code instrumentor of \iotguard.
As a result, we had to re-implement all modules of \iotguard for \oh from the scratch.

The purpose of code instrumentor module of \iotguard is to deploy hooks in an IoT application so that at runtime
(before any action is taken by the IoT application)
\iotguard's data collector can collect data and invoke security service to verify safety and security of taking an action.
Code instrumentor must be optimized so that if any application issues multiple actions in the same context, one common hook is deployed for all actions.
In our implementation of the code instrumentor, we implemented a parser that parses an IoT application of \oh platform and put those policy enforcing hooks before the action statement of the application.
Our parser ensures the optimized behavior of code instrumentor mentioned earlier.
%The output of the code instrumentor is the instrumented IoT applications.

%If any application issues multiple actions in the same context, one common hook is deployed for all actions.
%This optimization of using a common hook for multiple actions issued by an IoT application in the same context mentioned in the \iotguard paper.
%all applications communicate with \iotguard server before taking any action.
%To implement code instrumentor module of \iotguard for \oh, we implemented a parser that  vanilla IoT applications
%(IoT applications for \oh platform are called rules) 
%and generates instrumented IoT applications.

%Instrumented IoT applications implements a policy check before taking any action. If multiple actions are taken 

Data collector of \iotguard collects data (smart home events and actions) from instrumented IoT applications at runtime and stores the data in a dynamic model consists of states and transitions.
Each state represents an attribute of a device. Each transition from one state to another state represents the condition on which the state change has occurred. 
At a program level, a dynamic model is essentially a mutable directed graph where graph nodes are states and graph edges are transitions.
For example, consider an IoT app named light-control: ``\code{when motion-active after sunset, turn on light.}``
Upon receiving data from the instrumented \verb*|light-control| app data collector will create a dynamic model which has an event node \verb*|motion=ACTIVE|, an action node \verb*|Light=ON| and an edge with the condition ``\verb*|after| \verb*|sunset|''.

\begin{figure}
	\centering
	\begin{subfigure}[c]{0.35\linewidth}
%		\centering
		\includegraphics[width=\textwidth]{\FIGLOC/drm1.pdf}
%		\caption{\code{light-control}}
%		\vspace*{0.2in}
		\label{fig:light-app}
	\end{subfigure}
	\begin{subfigure}[c]{0.6\linewidth}
%		\centering
		\includegraphics[width=\textwidth]{\FIGLOC/drm2.pdf}
%		\vspace*{-0.2in}	
%		\caption{\code{heat-on}}
		\label{fig:combined-model}
	\end{subfigure}
	\vspace{-0.3in}
	\caption{ Dynamic model of a distinct IoT app and unified dynamic model of interacting IoT apps}
	\label{fig:dynamic-model}
\end{figure}
If multiple applications interact with each other, data collector will create a unified dynamic model.
For example: if \verb*|Light=ON| event from \verb*|light-control| app triggers \verb*|heat-on| app: ``\code{Turn on the heater and crockpot when light is on.}'', dynamic models of \verb*|light-control| app and \verb*|heat-on| app will be merged to generate a unified dynamic model. Dynamic model generated from mentioned IoT apps are presented in \FIGREF \ref{fig:dynamic-model}. In our implementation, we developed data collector module of \iotguard on top of python's Networkx library and we ensured our implementation has all the features of data collector module mentioned in the \iotguard paper.

%\section*{Policy Enforcement of IoTGuard}
Security service module of \iotguard reads safety and security policies, enforces those policies on the dynamic models generated by the data collector, and conveys the result of policy enforcement to the instrumented IoT application.
After processing (event, action) data received form the instrumented IoT application, data collector of \iotguard will add the corresponding nodes in the dynamic model and invoke security service to enforce policy on the dynamic model.
\iotguard supports three types of policies: General Policies, Application Specific Policies, Trigger-Action Specific Policies.
General policies are enforced directly on the dynamic model by applying the graph algorithms.
For example: A general policy mentioned in the \iotguard paper \cite{iotguard2019ndss}:
``\code{G3: An event handler of an app must not change a device attribute to a value which is used as an event in the event handler of  another app, e.g., door-lock event handler must not turn on a switch which is used in an event handler of an app that locks the door.}''
To enforce the policy \code{G3}, security service checked whether any dynamic model has a loop in it.
If there exists a loop in the dynamic model, security service module will deny the action by sending a deny response to the instrumented IoT app.

Application specific policies are written in a specific policy language mentioned in the \iotguard paper \cite{iotguard2019ndss}.
We implemented a policy parser to read policies written in \iotguard's policy language. To enforce application specific policy, our implementation of security service followed the description mentioned in \iotguard paper. % policy enforcement algorithm.
Each of the application specific policy is essentially a logical implication.
Hence, each policy has a premise condition and a conclusion condition.
A condition (premise/conclusion) is essentially a node in the dynamic model.
To enforce an application specific policy, security service tries to find a path(from the premise condition of the policy to the conclusion condition of the policy) in the dynamic model.
If there is a path in the dynamic model that matches with the corresponding path of a policy and if it is a ``deny'' policy, security service decides a policy is violated.
After the decision, security service removes the nodes that were added earlier by the data collector and it returns a deny response to the instrumented IoT application.

%In case of application specific policy, we were able to enforce only ``deny'' policy.
%In short, what we do is check if there is a path from the pre
%Although \iotguard's paper does not mention a specific policy enforcement algorithm, we crafted this policy enforcement algorithm from description of the paper.
%\input{\ALGOLOC/iotguardPolicyEnforcement}

To enforce trigger-action specific policy, we marked each of the physical device of the IoT testbed as Trusted and Secure and all virtual triggers (such as EmailSent, GoogleAssistantActivated) as Untrusted and Insecure. When enforcing trigger-action specific policy, security service will try to find a path from an untrusted(insecure) node to a trusted(secure) node.
After adding new nodes in the dynamic model, if there exists a path from untrusted node to a trusted node or a path from an insecure node to a secure node, security service will deny the actions and remove the corresponding nodes.




% a brief overview about the IoT defenses.
%Our goal to inform the reader about how policy enforcing IoT defenses ensures safety and security of IoT systems.
%Dynamic policy enforcing IoT defenses such as \expat, \patriot, and \iotguard enforces policy during execution of IoT applications.
%These IoT defenses consists of two primary components: Policy Enforcement Point (PEP) and Policy Decision Point(PDP). 
%Their policy enforcement mechanism works by interaction between the two components: Policy Enforcement Point and Policy Decision Point.
%For policy enforcement, all three of the IoT defenses rely on instrumenting IoT applications.

%PEP of all three IoT defenses works in a similar manner: by including necessary instructions to the source code of IoT applications.
%The process of including additional instructions is called instrumentation.
%By instrumenting IoT applications, IoT defenses guards actions of IoT applications with an \textbf{if} block, predicated on a function call to Policy Decision Point(PDP).
%With the function call to PDP, IoT application sends runtime information to PDP.
%This runtime information essentially include the context of the action being taken by the IoT application.
%
%PDP of IoT defense is responsible for making a decision about the impending action.
%IoT defenses make this decision based on user provided safety and security policies.
%User provides expected behavior of the system in the form of safety and security policy.
%Each of the IoT defense provides a policy language to help user in that process.
%PDP of IoT defense parse those policies and make a decision(allow/deny) about the impending action based on the policy semantics.
%
%Primary difference among these three IoT defenses(\expat, \patriot, and \iotguard) lies on the implementation of the policy decision component called PDP.
%PDP of \expat and \patriot is implemented on the instrumented IoT applications.
%As a result, instrumented applications in \expat and \patriot contains both PEP and PDP module.
%On the contrary, PDP of \iotguard is implemented on a local server as a separate component of the smart home system.
%PEP of \iotguard communicates with PDP of \iotguard through http requests.
%
%PDP of \expat and \patriot uses formal logic to reach a policy decision(allow/deny) about an impending action.
%%Both \expat and \patriot converts the policy to a logical expression and evaluates the logical expression to reach a policy decision.
%%\expat converts it's policy to QF-FOL.
%Upon receiving information about the impending action, \expat executes this action to reach a hypothetical system state.
%Then \expat validates that the hypothetical state is compliant with all policies.
%If the hypothetical system state is compliant with all policies, PDP of \expat reaches the decision to allow the action.
%For the opposite case, It denies the action.
%
%Upon receiving information about an impending action, \patriot checks whether impending action and the current system state is compliant with all policies.
%%\patriot converts it's policy to QF-MTL formula.
%%If impending action and the current system state is compliant, \patriot will allow the action to execute in real life.
%If the result of the compliance checking is true, PDP of \patriot will return allow response.
%Otherwise, it will return deny response. 
%
%PDP of \iotguard uses a different approach.
%Upon receiving events and actions from IoT applications, \iotguard's PDP creates a directed graph from those events and actions.
%%\iotguard's PDP maintains a directed graph which models the IoT applications.
%This graph is called dynamic rule model(DRM).
%PDP of \iotguard then does reachability analysis on the DRM to make a policy decision.
%%Each policy in \iotguard can be considered a path in the directed graph called DRM.
%Each policy in \iotguard essentially describes whether a path in the DRM is allowed or not.
%If PDP of \iotguard finds a path in the DRM, that is not allowed, it will send a deny response to PEP of \iotguard.
%For all other cases, PDP of \iotguard will return allow response.

%PDP of \iotguard tries to match the paths of relevant policies with paths in the DRM.
%Whenever, path of a policy matches with a path of DRM, PDP of \iotguard takes the policy decision according that matching policy.
%If the policy allows such a path in DRM, it will 


% that's all folks
\end{document}


